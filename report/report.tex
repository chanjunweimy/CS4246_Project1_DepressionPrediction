\documentclass{article}
\usepackage{aaai17}
\usepackage{amsmath}
\usepackage{times}
\usepackage{helvet}
\usepackage{courier}
\frenchspacing
\setlength{\pdfpagewidth}{8.5in}
\setlength{\pdfpageheight}{11in}

\title{
	CS4246 Project 1\\ Depression Prediction
}
\author{
	{\bf Team 01} \\
	Antoine Charles Vincent Garcia - A0159072A\\
	Chan Jun Wei - A0112084\\
	Chen Tze Cheng - A0112092\\
	Eric Ewe Yow Choong - A0112204\\
	Han Liang Wee, Eric - A0065517\\
	Ho Wei Li - A0094679\\
}

\begin{document}
 	\maketitle

	\begin{abstract}
	\begin{quote}
	Depression is a worrying issue in modern times. If left unregulated, it can be dentrimental both health and life. \\
	
	In this report, we illustrate the use of Gaussian Processes (GP) to calculate and model stress levels in society and with the data obtained, is used to estimate depression severity.
	\end{quote}
	\end{abstract}
	
	\section{1.	  Introduction}
	For our experiment, we will use the GP model to measure and compute depression severity via audio recordings. We will also be discussing about the desirable properties of the GP model as well as the technical details such as the GP model requirements for the proposed application and modifications made to enhance performance. We will also include our experimental evaluations and procedure in this report. \\

	The rationale of depression prediction is enable authorities to take appropriate actions if an area or individual is depressed. For example, suicide and crime are often linked to high depression and stress levels. The data can help authorities to monitor and mitigate crime in areas with marked as 'depressed'. In addition, annual health checks may include psychiatrist recommendations which is given to individuals who falls into the depression category. If successful, these data can even be used to break down depression into 'levels' which are a better of measurement.

	\section{2.  Gaussian Process Regression Model}
	As all individuals have varying inherent stress management, the use of the GP model for depression prediction makes use of all samples and feature information to perform the prediction including training data with different or uneven sampling rates. From the mean and variance obtained from previous data, we are able to predict if an individual is depressed. \\	

	\subsection{2.1  Qualitative Advantages}
	... \\

	\subsection{2.2  Important Requirements}
	 Our GP model requires multiple audio recordings of an individual's speech. \\

	\section{3.  Technical Approach}
	... \\	

	\section{4.  Evaluation}
	... \\

	\section{5.  Conclusion}	
	... \\

	\section{6.  Main Roles of Each Member}
	\begin{itemize}
		\item \textbf{Antoine Charles Vincent Garcia}: 
		Scripting the program, setting up machine learning libraries and running tests.
		\item \textbf{Chan Jun Wei}: 
		Project technicalities such as problem formulation and modelling, mathematics and experiment planning.
		\item \textbf{Chen Tze Cheng}: 
		Project technicalities such as problem formulation and modelling, mathematics and experiment planning.
		\item \textbf{Eric Ewe Yow Choong}: 
		Documentation especially writing of the motivation, recording research findings and keeping track of requirements.
		\item \textbf{Han Liang Wee, Eric}: 
		Scripting the program, setting up machine learning libraries and running tests.
		\item \textbf{Ho Wei Li}: 
		Documentation especially writing up the motivation, recording research findings and keeping track of requirements.
	\end{itemize}
	
	\section{References}
	\begin{itemize}
		\item
		[1] Michel Valstar, Jonathan Gratch, Bjorn Schuller, Fabian Ringeval, Denis Lalanne, Mercedes Torres Torres, Stefan Scherer, Giota Stratou, Roddy Cowie, Maja Pantic, ``AVEC 2016 - Depression, Mood, and Emotion Recognition Workshop and Challenge", MAY. 27, 2016 \\
	\end{itemize}

\end{document}