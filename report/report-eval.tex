\documentclass{article}
\usepackage{aaai17}
\usepackage{amsmath}
\frenchspacing
\setlength{\pdfpagewidth}{8.5in}
\setlength{\pdfpageheight}{11in}

\title{
	CS4246 Project 1\\ Depression Prediction
}
\author{
	{\bf Team 01} \\
	Antoine Charles Vincent Garcia - A0159072A\\
	Chan Jun Wei - A0112084\\
	Chen Tze Cheng - A0112092\\
	Eric Ewe Yow Choong - A0112204\\
	Han Liang Wee, Eric - A0065517\\
	Ho Wei Li - A0094679\\
}

\begin{document}
 	\maketitle

	\begin{abstract}
	\begin{quote}
	In this report, we illustrate the use of Gaussian Processes to calculate and model stress levels in society and with the data obtained, is used to estimate depression severity.
	\end{quote}
	\end{abstract}
	
	\section{Introduction}

	\section{Gaussian Process Regression Model}

	\section{Technical Approach}

	\section{Evaluation}
	In order to test our Gaussian Process model, we conducted tests on data obtained from Audio/Visual Emotion Challenge and Workshop(AVEC 2016). The goal of AEVC is to weigh-in on the various approaches(visual, audio) used to recognize emotions under unambiguous conditions. AVEC 2016 provided 2 pieces of data as input: visual and auditory data. However, we would be reducing the scope of the experiment, limiting the experiment to only the auditory data. Two Sub-Challenges are lised in AVEC 2016. We are only interested in the Depression Classification Sub-Challenge, which requires participants to classify inputs by the PHQ-8 score.

		\subsection{Data}
		The depression data used in AVEC 2016 was obtained from the benchmarking database, the Distress Analysis Interview Corpus - Wizard of Oz(DAIC-WOZ). Data collected from DAIC-WOZ include audio and video recordings and the corresponsing PHQ-8 score, which is a frequently used self-report scheme to access severity of depression{CITE}. Henceforth, we would need to pre-process the auditory data before we use it in our Gaussian Process Model. The data is pre-processed as described in the section above. The distribution of the depression severity scores in both training and development set is given in FIG{}. 

cite{DBLP:journals/corr/ValstarGSRLTSSC16} 
	\section{Conclusion}	

	\section{Main Roles of Each Member}
	\begin{itemize}
		\item \textbf{Antoine Charles Vincent Garcia}: 
		Scripting the program, setting up machine learning libraries and running tests.
		\item \textbf{Chan Jun Wei}: 
		Project technicalities such as problem formulation and modelling, mathematics and experiment planning.
		\item \textbf{Chen Tze Cheng}: 
		Project technicalities such as problem formulation and modelling, mathematics and experiment planning.
		\item \textbf{Eric Ewe Yow Choong}: 
		Documentation especially writing of the motivation, recording research findings and keeping track of requirements.
		\item \textbf{Han Liang Wee, Eric}: 
		Scripting the program, setting up machine learning libraries and running tests.
		\item \textbf{Ho Wei Li}: 
		Documentation especially writing up the motivation, recording research findings and keeping track of requirements.
	\end{itemize}
	
	\section{References}

\end{document}
